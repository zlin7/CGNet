\documentclass[12pt]{amsart}

\usepackage{amsmath}
\usepackage{amssymb}
\usepackage{enumitem}
\usepackage{graphicx}
\usepackage{mathtools}
\usepackage{booktabs}
\usepackage{hyperref}
\usepackage{algorithm}
\usepackage[noend]{algpseudocode}
\usepackage[margin=0.5in]{geometry}
\renewcommand{\arraystretch}{1.5}
\graphicspath{ {images/} }

\DeclarePairedDelimiter\ceil{\lceil}{\rceil}
\DeclarePairedDelimiter\floor{\lfloor}{\rfloor}

\theoremstyle{plain}
\newtheorem{Thm}{Theorem}
\newtheorem{Cor}{Corollary}
\newtheorem{Main}{Main Theorem}
\renewcommand{\theMain}{}
\newtheorem{Lem}{Lemma}
\newtheorem{Prop}{Proposition}

\theoremstyle{definition}
\newtheorem{Def}{Definition}
\newtheorem{Exer}{Exercises}
\newtheorem{Prob}{Problem}

\theoremstyle{remark}
\newtheorem{notation}{Notation}
\renewcommand{\thenotation}{}

\errorcontextlines=0
\numberwithin{equation}{section}

\newcommand\independent{\protect\mathpalette{\protect\independenT}{\perp}}
\def\independenT#1#2{\mathrel{\rlap{$#1#2$}\mkern2mu{#1#2}}} 
\makeatletter
\def\BState{\State\hskip-\ALG@thistlm}
\makeatother
\begin{document}

%Topmatter
\title{Testing Clebsch-Gordan Coefficients}
\date{\today}  %% Changed to date
%End topmatter
\maketitle
\paragraph{\textbf{Wigner Matrix}}
\begin{enumerate}
    \item Using the \texttt{lie\_learn} library: in \texttt{https://github.com/AMLab-Amsterdam/lie\_learn/}, under 
    \begin{verbatim}
    lie_learn - representations - SO3 - wigner_d.py
    \end{verbatim}
    \item Using the table in \texttt{https://link.springer.com/content/pdf/bbm\%3A978-1-4684-0208-7\%2F1.pdf} (for rank 1 and 2 only)
\end{enumerate}

\paragraph{\textbf{Equations to test}}
\begin{enumerate}
    \item 
    \begin{align*}
        \sum_{m_1', m_2'} C^{(l_0,m_0)}_{(l_1,m_1')(l_2,m_2')} D^{(l_1)}_{m_1',m_1}(g)D^{(l_2)}_{m_2',m_2}(g) = \sum_{m_0'} D^{(l)}_{m_0,m_0'}(g) C^{(l_0,m_0')}_{(l_1,m_1)(l_2,m_2)}
    \end{align*}
    (\texttt{https://arxiv.org/pdf/1802.08219.pdf})
    
    \item Equation (A.7)
    \begin{align*}
        D^{l_1}_{m_1', m_1}(\Omega)D^{l_2}_{m_2',m_2}(\Omega) = \sum_{l, m, m'} C(l_1,l_2,l;m_1',m_2', m') C(l_1, l_2, l;m_1, m_2, m)D^l_{m',m}(\Omega)
    \end{align*}
    (\texttt{https://link.springer.com/content/pdf/bbm\%3A978-1-4684-0208-7\%2F1.pdf})
    
    \item 
    \begin{align*}
        D^l_{mn} = \sum_{m_1+m_2=n,n_1+n_2=n} D^{l_1}_{m_1 n_1} D^{l_2}_{m_2 n_2} C(l_1,l_2,l;m_1,m_2,m)C(l_1,l_2,l;n_1,n_2,n)
    \end{align*}
    (page 24 of \texttt{http://citeseerx.ist.psu.edu/\break viewdoc/download?doi=10.1.1.152.9702\&rep=rep1\&type=pdf})

\end{enumerate}

\paragraph{\textbf{Current goal}}: Verify the three equations using the two way of computing wigner matrices.

\end{document}

